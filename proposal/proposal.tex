% Original Latex template author:
% Frits Wenneker (http://www.howtotex.com)
% License:
% CC BY-NC-SA 3.0 (http://creativecommons.org/licenses/by-nc-sa/3.0/)
%%%%%%%%%%%%%%%%%%%%%%%%%%%%%%%%%%%%%%%%%

%----------------------------------------------------------------------------------------
%	PACKAGES AND OTHER DOCUMENT CONFIGURATIONS
%----------------------------------------------------------------------------------------

\documentclass[paper=a4, fontsize=11pt]{scrartcl} % A4 paper and 11pt font size

\usepackage[T1]{fontenc} % Use 8-bit encoding that has 256 glyphs
\usepackage{fourier} % Use the Adobe Utopia font for the document - comment this line to return to the LaTeX default
\usepackage[english]{babel} % English language/hyphenation
\usepackage{amsmath,amsfonts,amsthm,enumerate} % Math packages

\usepackage{graphicx}
\usepackage[colorlinks=true,linkcolor=blue]{hyperref}
\usepackage{sectsty} % Allows customizing section commands
\allsectionsfont{\centering \normalfont\scshape} % Make all sections centered, the default font and small caps

\usepackage{fancyhdr} % Custom headers and footers
\pagestyle{fancyplain} % Makes all pages in the document conform to the custom headers and footers
\fancyhead{} % No page header - if you want one, create it in the same way as the footers below
\fancyfoot[L]{} % Empty left footer
\fancyfoot[C]{} % Empty center footer
\fancyfoot[R]{\thepage} % Page numbering for right footer
\renewcommand{\headrulewidth}{0pt} % Remove header underlines
\renewcommand{\footrulewidth}{0pt} % Remove footer underlines
\setlength{\headheight}{13.6pt} % Customize the height of the header

\numberwithin{equation}{section} % Number equations within sections (i.e. 1.1, 1.2, 2.1, 2.2 instead of 1, 2, 3, 4)
\numberwithin{figure}{section} % Number figures within sections (i.e. 1.1, 1.2, 2.1, 2.2 instead of 1, 2, 3, 4)
\numberwithin{table}{section} % Number tables within sections (i.e. 1.1, 1.2, 2.1, 2.2 instead of 1, 2, 3, 4)

\setlength\parindent{0pt} % Removes all indentation from paragraphs - comment this line for an assignment with lots of text

\DeclareMathOperator*{\argmin}{\arg\!\min}
\DeclareMathOperator*{\argmax}{\arg\!\max}
\usepackage{lettrine,setspace}
\usepackage[margin=.8in]{geometry}

%----------------------------------------------------------------------------------------
%	TITLE SECTION
%----------------------------------------------------------------------------------------

\newcommand{\horrule}[1]{\rule{\linewidth}{#1}} % Create horizontal rule command with 1 argument of height

\title{	
\normalfont \normalsize 
\textsc{Arbitrary Labs} \\ [5pt]
\horrule{0.5pt} \\[0.4cm] % Thin top horizontal rule
\huge EE 123 Project Proposal \\ % The assignment title
\horrule{2pt} \\[-0.1cm] % Thick bottom horizontal rule
}

\author{Rishi Sharma\\ Wisam Reid}

\begin{document}
	
\maketitle % Print the title

%----------------------------------------------------------------------------------------
%	BEGIN DOCUMENT
%----------------------------------------------------------------------------------------
We have two potential project proposals, both of which we find interesting.\\\\

{\Large 1. \indent Sparse Reconstruction of Sound Fields}\\

Not unexpectedly, methods for reconstructing sound fields to create the perception of ``spatialized'' audio simplify to solving a least squares problem. Our sound spatialization system (we call it N-Dimensional Audio) currently uses Higher Order Ambisonics (HOA), which expresses a sound field as a sum of simple functions in spherical coordinates known as spherical harmonics. A discrete array of speakers can only produce an approximation to this sound field, which it will optimize according to some loss function. We propose to consider and implement an algorithm for sound field reconstruction based on the notion that the sound field is sparse, and leverage the work in compressed sensing to do so. We could use some of these papers as our guide: \href{http://ieeexplore.ieee.org/stamp/stamp.jsp?tp=&arnumber=5346520}{1},  \href{http://ieeexplore.ieee.org/stamp/stamp.jsp?arnumber=05946441}{2},   \href{http://spincom.umn.edu/files/pdfs/tasl2010nov.pdf}{3}.\\

We feel there is a good chance that you do not approve this project because it is not focused on the type of DSP we covered in this class and also, if we fail theoretically in some regard, we may end up with a project which has almost no physical deliverables, which may not be acceptable for a class like EE123. Regardless, if you all willing to allow us to pursue this (and maybe give us some small help along the way), we think we can be successful with this project.\\\\

{\Large 2. \indent Decomposition of Stereo into Independent Components for Spatialized Reconstruction}\\

We had a great idea (partially inspired by Miki) and Googled it and found this very helpful \href{http://theses.eurasip.org/media/theses/documents/cobos-maximo-application-of-sound-source-separation-methods-to-advanced-spatial-audio-systems.pdf}{PhD Thesis} that had effectively done {\it exactly} what we were thinking about. The problem of source separation from two channels of music ultimately also comes down to an undetermined system (at least in the formulation presented in this thesis, other models differ, most notably Independent Component Analysis (ICA)), but we are of the belief that there is far more information that can be extracted from the spectrogram (and variants) than has been done so far. We believe there is more than enough information in two channels of stereo to uniquely reconstruct the original stems which make up the song (given some assumptions on the signal which will likely hold for any type of music that sounds even remotely reasonable to the human ear i.e. not like Gaussian noise). We have no intention to theoretically analyze any of these properties, but rather leverage this intuition to do very good source separation and reproduce music in a cool new way.




\end{document}